\chapter{Conclusion}

Tout d'abord, nous avons pris du temps à démarrer. Bien que nous nous soyons rapidement mis d'accord sur les technologies à utiliser, nous avons pris beaucoup de temps à concevoir la base de données.\\

Pour 2 raisons: tous d'abords, il y a plusieurs fichier, et en particulier "Inscription", qui contiennent beaucoup de colonnes.\\

Il fallait donc être rigoureux pour ne rien oublier, mais tout en veillant à n'avoir que des tables et des attributs pertinents. En effet,
il fallait respecter la contrainte de normalisation de la table, et éviter aux maximum les redondances.\\

De plus, il y a globalement de nombreux fichiers, et de nombreuses données dans chacun d'eux. Tout était plus ou moins liés, ce qui augmentait la complexité.
De plus, cette intrication entre les fichiers et les tables rendaient plus difficile de travailler à plusieurs simultanément sur le schéma de la base de données.\\


Pour l'application, nous avons bien réussi à travailler ensemble et à avoir un code aussi propre que possible. Cela est passé notamment en ayant de nombreux fichiers différents pour séparer les différents domaines de l'application. Nous avons de plus veillé à créer des fonctions simples et assez courtes ne faisant qu'une chose à la fois. Nous les avons documentées, afin de savoir ce qu'elle font, prennent en argument et renvoie. Cela à permis de facilement travailler à plusieurs sur la même chose en parallèle, et de pouvoir comprendre et corriger le code des autres avec aisance.\\


Pour la gestion de projet, les différents documents fait au début du projet, et complété éventuellement par la suite, ont permis de savoir quoi faire. 
Nous avons ainsi évité de nous disperser sur des idées ou des conceptions différentes.\\

De plus, nous étions tous les 3 à l'aise avec la méthodologies que nous avons suivi, consistant à travailler de notre côté.
Nous mettions en commun notre avancé et nos problèmes lors de réunion hebdomadaire de généralement 30 à 45min. Elles ont rarement dépassé 1h, lors de réflexions et travails en commun, notamment pour des documents de gestion de projet ou modification de la base de données.




\section{Conclusion personnelle }
\subsection{LEDEUX Flavien}
J'ai trouvé ce projet intéressant car les informations vis à vis du projet venaient petit à petit et que le sujet restait vague mais laissant la possibilité d'être précisé à l'aide de questions, ce qui est quelque chose qui ressemble à un exemple concret de projet que l'on risque de rencontrer dans le futur.

D'autre part, ce projet m'a permit de me familiariser avec un langage qui m'était presque inconnu : le langage des fichiers batch ainsi que de revoir les scripts bash que j'avais vue au tout début de ma formation d'IUT.

De plus le projet m'a permis de travailler avec des personnes qui m'étaient inconnues confrontant ainsi mes méthodes de travailles à celles des autres forçant ainsi une adaptation enrichissante dans le but de garder le meilleur de chacun.

Je trouve cependant que notre projet n'as malheureusement pas atteint toutes les attentes que j'avais vis à vis de ce dernier. Cela est pour de multiples raisons telles que le manque de temps pour approfondir certaines thématiques, la présence des partiels lors de la fin de projet limitant le temps disponible a consacrer a ce dernier et un investissement plus important possible vis à vis du projet.   

Si je devais refaire le projet de zéro, j'essaierais de mieux gérer la l'utilisation de notre temps en nous planifiant mieux et en séparant mieux les différentes tâches. 
De plus, j'essaierais aussi de nous faire travailler avec une cadence plus importante sur le début du projet.

D'autre part, la chose que j'essaierais le plus d'appliquer à d'autres projets est notre méthode rigoureuse de planification et d'exécution des réunions qui nous a permis d'éviter de rester bloquer trop longtemps.
D'autre part, le contenu des réunions était toujours pertinent et le travail était au rendez-vous lors de ces dernières ce qui ce reflète dans des temps de réunion ne dépassant qu'occasionnellement une heure.

Enfin, ce projet m'as forcé a prendre du recul vis a vis de ma méthode de travailler car j'ai du développer un rôle plus en retrait vis à vis de la création de la base de données pour pouvoir laisser mes collègues gagner de l'expérience avec cette notion au lieu de me concentrer la faire a ma façon au détriment de mes collègues.

Pour conclure, j'ai trouvé ce projet enrichissant bien qu'il laisse un arrière goût amer en quand on réfléchit à ce qui aurait pue être amélioré où rajouter.


\subsection{ROURA Guilhèm} Ce projet fut mon premier projet complet, allant de la conception à la réalisation.
J'ai d'abord pu m'entraîner et m'améliorer à la conception d'une base de données, chose que je ne maîtrisais pas forcément au départ.
J'ai pu bien prendre conscience et mettre e pratique des concepts tel que la normalisation d'une base de données, ou encore la manière d'éviter la redondance par des tables auxiliaires.

Pour ce qui est de la programmation, cela ma permis de faire mon premier projet complet en python.
Je me sentais déjà assez à l'aise avec, mais je n'avais jamais fait de projet de cette ampleur.

De manière général, j'ai pu me rendre compte de la difficulté et la complexité d'entière concevoir une application qui soit solution à un problème précis.
La liberté de conception permet de se sentir réellement impliqué et de faire des choix en fonction de ses préférences et de ce qu'on juge le plus pertinent.

Cette liberté est cependant à double tranchant: Il faut savoir se fixer des limites pour ne pas rester bloqué sur la phase de conception, tout en étant aussi rigoureux que possible. 
Tous les mauvais choix auront des répercussions par la suite, ils peuvent obliger à s'adapter lors de la phase de réalisation, par exemple en raison du langage et des technologies utilisées.
Mais la conception même peut être à revoir, par exemple pour le schéma de la base de données, si cette dernière ne répond finalement pas à toutes les problématiques posés.
On peut aussi la revoir si une solution bien meilleur est trouvée par la suite.


Enfin, ce projet fut le premier que je fais avec des personnes que je ne connais pas, et ayant des parcours différents.
Ces différences m'ont cependant semblé être plus des forces que des faiblesses, puisque chacun à ses propres points forts et connaissance à apporter.
Le fait de devoir adapter sa méthode de travail m'a semblé valoir le coup, puisqu'on peut tenter de garder le meilleur de chacun en terme de méthode.
De plus, chacun s'est investi et s'est senti responsable du projet, dans l'échec comme dans la réussite, ce qui constitue une bonne synergie, et nous a permit de rester motivé. 


Pour finir, mon ressenti sur le projet est que le temps pris au début pour concevoir la base de données et commencer les scripts python fut trop important.
Cela à laissé moins de temps pour avoir un résultat satisfaisant.
Cependant, la grande complexité des données à analyser a obligé à prendre du temps pour bien tout comprendre et tout concevoir correctement.

\subsection{ZHU Céline}

La différence entre théorie et pratique m’a parfois surprise durant la réalisation de ce projet que ce soit en gestion de projet ou en informatique.\\ L'approfondissement du Flask par exemple a été une expérience enrichissante allant au delà des cours fournis et allant dans le domaine de la recherche personnelle. C'était toujours très gratifiant de faire réussir à coder un nouveau bout du site et d'interagir très rapidement avec.\\

La réalisation de mon premier projet complet ne vient cependant pas sans défaut. En effet, nous avons eu la chance d'avoir un sujet qui nous laissez beaucoup de liberté et parfois ne maîtrisant pas bien mes limites, je me lançait dans des projets trop ambitieux pour mon niveau: par exemple le carousel qui a été très chronophage pour un rendu finalement qui n'était peut-être pas tellement intéressant.\\

C'est aussi mon premier projet en méthode Agile avec des gens que je ne connaissais pas. Heureusement, nous avions eu des complémentaires grâce à des parcours scolaires très différents nous permettant ainsi sur beaucoup de niveau de nous combler nos lacunes.\\

Qu’est-ce qu’on pourrait améliorer ? Beaucoup de choses, mais je citerai trois
points clés : un démarrage en gestion de projet plus rapide, une sauvegarde plus automatique des fonctions codées et enfin être moins optimiste quant aux dates buttoirs. Ce premier projet d’informatique m’a permis de vérifier la bonne acquisitions des compétences acquise durant cette première année à TELECOM Nancy.En conclusion, ce fut une expérience enrichissante qui s’est très bien passée dû à la synergie de travail du groupe.