\title{P2II 1A - Compte-rendu n°1}
\begin{center}
        \begin{tabular}{|l|l|}
        \hline
   
\textbf{Motif de la réunion :}&\textbf{Lieu de la réunion :}\\

Réunion de démarrage du projet&Réunion en visioconférence sur Discord\\


        \hline
        \textbf{Présents :}&\\

Flavien LEDEUX&\textbf{Date :} 03/04/2021\\

Guilhèm ROURA&\\
Céline ZHU&\textbf{Heure :} 10h01\\

&\\
\textbf{Absent :}&\textbf{Durée :} 45 minutes\\

Personne&\\
&\\
\textbf{Rédaction du compte-rendu :}&\\

Céline ZHU&\\

        \hline
        \end{tabular}\\
        \end{center}
\minitoc


\section{Prendre connaissance du sujet}

Rapport sur la Gdp et la réflexion\\
    Objets à rendre:\\
    Base de données SQL standard + schéma\\
    Rapports formats html ( rédigé en fonction de quoi à demander à M Da silva ou M Oster)\\
Afficher les résultats des requêtes que ce soit en schéma ou par texte\\
L’état des fichiers final ou pas.\\
\section{Établir les outils de GdP}
Fichier Readme.md\\
Matrice RACI\\
Gantt \\
SWOT\\
\section{Établir le Cahier des charges}
Base de données SQL:\\
script Python rendant une base de données cohérente\\
Avec des informations\\
Trois étapes: Design, construction, tests boîte noire/blanche\\
	  - Application Python ( html):\\
			- Tous Flask\\
			- Définir tous les liens qui sont liés à notre application\\
			- Définir les conventions Html\\
			- Fiches de style\\
\section{Latex (?)}
Création du latex sur Overleaf et séparation en plusieurs fichiers Tex plus tard.\\
Programmation\\
 Flavien:  Python(flask) et Sqlite sont plus pratiques pour des amateurs. et on peut s’aider de la WebBD.\\

\section{ Planifier la semaine d'après }
	- Faire les scripts de la base de données pour les créer et le schéma.\\
		- Schéma: Flavien/Guilhèm\\
		- Scripts: Céline\\
		- Tests: Flavien/Céline\\
	-Commencer à réaliser les classes représentant les différentes éléments à ajouter dans la BDD.\\
	- Documenter vos actions en latex.\\
Prochaine réunion:\\
Samedi à 10h01 10/04/21:\\
	odj: \\
- Avancement du projet\\
	- Construction de la matrice RACI\\
	- Faire le Gantt\\
	- Dire les endroits qui nous ont posés problèmes\\
